% arara: pdflatex
% arara: biber
% arara: pdflatex
% arara: pdflatex

\documentclass[a4paper,10pt,titlepage]{article}

\usepackage[utf8]{inputenc}
\usepackage{subcaption}
\usepackage[czech]{babel}
\usepackage{fontenc}
\usepackage{graphicx}
\usepackage[backend=biber, style=iso-numeric, urldate=comp]{biblatex}
\usepackage[pdftex]{hyperref}
\usepackage{url}
\usepackage{csquotes}
\usepackage[parfill]{parskip}
\usepackage[version=4]{mhchem}
\usepackage{siunitx}
\usepackage{titlesec}
\usepackage{float}
\usepackage{circuitikz}
\usepackage{listings}
\usepackage{pgfplots}

\addbibresource{luxmetr.bib}

\DefineBibliographyStrings{czech}{
  url         = [Online]\addspace Dostupné z~,
}

\counterwithin{figure}{section}
\hyphenation{zig-bee}
\sisetup{detect-all}
\setcounter{secnumdepth}{4}

\titleformat{\paragraph}
{\normalfont\normalsize\bfseries}{\theparagraph}{1em}{}
\titlespacing*{\paragraph}
{0pt}{3.25ex plus 1ex minus .2ex}{1.5ex plus .2ex}

\author{Tomáš Kysela}
\title{Luxmetr}
\date{\parbox{\linewidth}{\centering%
  \today\endgraf\bigskip
 Vytvořeno jako semestrální projekt pro předmět BAB34BSP\endgraf\medskip
 FEL ČVUT}}



\begin{document}

\pagenumbering{Alph}
\begin{titlepage}

	\maketitle

	\thispagestyle{empty}
\end{titlepage}

\pagenumbering{arabic}

\tableofcontents
\listoffigures
\listoftables

\newpage

\section{Úvod} \label{sec:intro}
Cílem projektu je vytvořit funkční jednoduchý luxmetr za využítí fotorezistoru, zprácování výstupu v mikrokontroleru a zobrazení na displej. Dále je tento senzor kalibrován za využití komerčního přístroje.

\section{Teoretický úvod} \label{sec:theory}
Elektromagnetické záření je pohyb částic vyvolaných kmitáním nabitých částic, či pohybem nabitých částic ve vysokých rychlestech. Samotné světlo je poté elektromagnetické záření přibližně o vlnových délkách 380 až 740nm.~\cite{radiation}

\begin{figure}[h]
	\centering
	\includegraphics[width=0.5\textwidth]{assets/spectrum.png}
	\caption[Spektrum elektromagnetického záření]{Spektrum elektromagnetického záření~\cite{radiation}}
\end{figure}

\subsection{Jednotky světelné intenzity}
\textit{Lumen} je jednotka světelného toku, definovaná jako světelný tok vyzařovaný do prostorového úhlu 1 steradiánu bodovým zdrojem, jehož svítivost je 1 kandela. Pokud tuto jednotku dáme na metr čtvereční dostáváme \textit{lux}. Běžné hodnoty luxů jsou dle tabulky~\ref{tab:lux_typical}.

\begin{table}[h]
 \centering
 \begin{tabular}{c | c}
  Světelné podmínky & Typické hodnoty lux \\ \hline \hline
  Jasný den & 10000 až 25000 \\ \hline
  Zatažený den & 1000 až 5000 \\ \hline
  Umělé vnitřní osvětlení & 100 až 500 \\ \hline
  Měsíční světlo & 1
 \end{tabular}
  \caption{Typické hodnoty lux}
  \label{tab:lux_typical}
\end{table}

\subsection{Fotorezistor}
Princip fotorezistoru (LDR) záleží na fotoelektrickém jevu. Foton narazí do elektronu ve valenční sféře a předá mu svoji energii, tím elektron získá dostatek energie k překonání zakázaného pásu a skočí do vodivé vrstvy. Takto uvolněné elektrony poté přispívají ke snížení elektrické odporu ve fotorezistoru. Tedy čím více světla dopadá na fotorezistor tím menší je jeho odpor.

\section{Návrh luxmetru}
Luxmetr realizujeme za využití LDR. Ten zapojíme jako napěťový dělič s pevným odporem a následujeme sledujeme úbytek napětí na LDR. K tomu využíváme Analog-To-Digital Convertor. Přečtené hodnoty napětí následně pomocí funkce získané z kalibrace za pomocí curve-fitting přepočítáme na hodnotu lux.

\subsection{Hardware zapojení}
\begin{figure}[h]
\centering
\resizebox{1\textwidth}{!}{%
\begin{circuitikz}
\tikzstyle{every node}=[font=\LARGE]
\draw  (3.75,12.5) rectangle  node {\LARGE LDR} (6.25,10);
\draw  (8.75,12.5) rectangle  node {\LARGE ADC} (11.25,10);
\draw  (13.75,12.5) rectangle  node {\LARGE MCU} (16.25,10);
\draw  (18.75,12.5) rectangle  node {\LARGE LCD} (21.25,10);
\draw [->, >=Stealth] (6.25,11.25) .. controls (7.5,11.25) and (7.5,11.25) .. (8.75,11.25);
\draw [->, >=Stealth] (11.25,11.25) .. controls (12.5,11.25) and (12.5,11.25) .. (13.75,11.25);
\draw [->, >=Stealth] (16.25,11.25) .. controls (17.5,11.25) and (17.5,11.25) .. (18.75,11.25);
\end{circuitikz}
}%
\caption{Logické zapojení prvků}
\label{fig:my_label}
\end{figure}


\begin{figure}[h]
 \centering
  \begin{subfigure}{0.4\textwidth}
   \centering
   \includegraphics[height=10cm]{assets/ldr.png}
   \caption{Zapojení fotorezistoru}
  \end{subfigure}
  \begin{subfigure}{0.4\textwidth}
   \centering
   \includegraphics[height=10cm]{assets/lcd.png}
   \caption{Zapojení LCD}
  \end{subfigure}
  \caption{Zapojení jednotlivých prvků projektu}
\end{figure}

\begin{table}[h]
 \centering
 \begin{tabular}{c || c | c }
  Součástka & počet & cena/kus \\ \hline \hline
  STM32 Nucleo-G441RE & 1 & 363 Kč \\ \hline
  TRIMMER 64 P 10K CN 250ppm & 1 & 14 Kč \\ \hline
  fotorezistor GL5516 & 1 & 7 Kč \\ \hline
  rezistor 5.1 \si{\kilo\ohm} & 1 & 4 Kč \\ \hline
  rezistor 470 \si{\ohm} & 1 & 4 Kč \\ \hline
  DM1602AB alfanumerický LCD & 1 & 85 Kč
 \end{tabular}
  \caption{Seznam použitých součástek}
\end{table}


\subsection{Software zpracování}
V software vyžíváme předgenerované prostředí z STM32 CUBE IDE. Následně doplníme logiku kde pravidelně spouštíme ADC konverzi, následně přečteme hodnotu, převedeme na napětí a z napětí přepočítáme na luxy, který následně zobrazíme.

\begin{lstlisting}[language=c, breaklines=true]
 int main(void)
{
  char txt[30];
  uint16_t adc_raw;
  uint32_t lux;
  float voltage;

  Lcd_PortType ports[] = {D4_GPIO_Port, D5_GPIO_Port, D6_GPIO_Port, D7_GPIO_Port};
  Lcd_PinType pins[] = {D4_Pin, D5_Pin, D6_Pin, D7_Pin};

  Lcd_HandleTypeDef lcd = Lcd_create(ports, pins, RS_GPIO_Port, RS_Pin, EN_GPIO_Port, EN_Pin, LCD_4_BIT_MODE);

  while (1)
  {
	HAL_ADC_Start(&hadc1);
	HAL_ADC_PollForConversion(&hadc1, 100);
	adc_raw = HAL_ADC_GetValue(&hadc1);
	HAL_ADC_Stop(&hadc1);

	voltage = ((double) adc_raw / 4096) * 3.3;

	lux = LUX_SCALAR * pow(voltage, LUX_EXPONENT);

	sprintf(txt, "%ld lux", lux);
	Lcd_clear(&lcd);
	Lcd_cursor(&lcd, 0, 0);
	Lcd_string(&lcd, txt);

	HAL_GPIO_TogglePin(LD2_GPIO_Port, LD2_Pin);
	HAL_Delay(500);
  }
}
\end{lstlisting}

\subsection{Curve fitting}
Vzhledem k naměřeným hodnotám jsme vypozorovali přibližně logaritmickou závislost. Tedy jsme jak naměřené hodnoty, tak napětí zlogaritmovali dekadickým logaritmem, tedy jsme dostali rovnici \ref{eq:log}, kde $y$ jsou naměřené luxy a $x$ naměřené napětí.
\begin{equation}
 \label{eq:log}
 \log_{10}{y} = m \log_{10}{x} + c
\end{equation}

Tu jsme následně aproximovali a následně převedli na rovnci \ref{eq:fin}.
\begin{equation}
 \label{eq:fin}
 y = x^m \cdot 10^c = x^m \cdot a
\end{equation}
Výsledné hodnoty jsou poté $m = -1.974$ a $a = 48.842$.

\begin{figure}[h]
 \centering
\begin{tikzpicture}
 \begin{axis}[
    axis lines = left,
    xlabel = {Napětí [\si{\volt}]},
    ylabel = {Světelný tok [\si{\lux}]},
    samples=100,
    xmin=0,
    xmax=4,
    ymin=0,
    ymax=15000,
]
  \addplot[color=red,domain=0.01:4]{48.824 * x ^ (-1.974)};
  \addlegendentry{Aproximovaná funkce}

  \addplot[color=blue,mark=x, only marks] coordinates {
  (0.02, 12800)
  (0.05, 13500)
  (0.13, 11870)
  (0.45, 731)
  (0.74, 321)
  (0.82, 301)
  (0.83, 285)
  (1.49, 36)
  (1.5, 36)
  (1.53, 35)
  (2.41, 2)
  (2.42, 1)
  (2.42, 1)
  };
  \addlegendentry{Naměřené hodnoty}
 \end{axis}

\end{tikzpicture}

\end{figure}

\section{Závěr}
Úspěšně jsme sestavili luxmetr, kalibrovali ho na rozsahu 0 až 14000 lx. Výsledné hodnoty úspěšně zobrazujeme na LCD alfanumerickém displeji.
\section{Zdroje}
\nocite{*}
\printbibliography[heading=none]
\end{document}
